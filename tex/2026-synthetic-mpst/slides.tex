\DocumentMetadata{
  lang        = en-UK,
  pdfstandard = ua-2,
  pdfstandard = a-4f, %or a-4
  tagging=off,
  tagging-setup={math/setup=mathml-SE}
}
\documentclass{ltx-talk}

\usepackage[utf8]{inputenc}
\usepackage[T1]{fontenc}
\usepackage{lmodern}
\usepackage[dvipsnames]{xcolor}
\usepackage{pagecolor}
\usepackage{alltt}



%%%%%%%%%%%%%%%%%%%%%%%%%%%%%%%%%%%%%%%%%%%%%%%%%%%%%%%%%%%%%%
%% TikZ libraries
\usepackage{tikz}
\usetikzlibrary{automata, positioning, arrows}

\usetikzlibrary{calc,decorations.pathreplacing,
  decorations.pathmorphing,
  decorations.markings,
  shapes, fit, shadows.blur,
  arrows.meta, tikzmark
  }
\usetikzlibrary{cd}
\usetikzlibrary{positioning}
\usetikzlibrary{arrows}
\tikzset{
    arrow at end/.style={
        decorate,decoration={
            markings,
            mark=at position .999 with{
                \arrow{#1};
            }
        }
    }
}
\tikzstyle{highlight}=[fill=red,opacity=0.5,ultra thick]
\tikzset{
  onslide/.code args={<#1>#2}{%
  \only<#1>{\pgfkeysalso{#2}}}
}

\usepackage[framemethod=TikZ]{mdframed}

%%%%%%%%%%%%%%%%%%%%%%%%%%%%%%%%%%%%%%%%%%%%%%%%%%%%%%%%%%%%%%

\usepackage[skins]{tcolorbox}

\newtcolorbox{sticky}[1][]
{%
  enhanced,
  center upper,
  %fontupper=\strut,
  drop fuzzy shadow southeast,
  boxrule=0pt,
  %sharp corners,
  colframe=yellow!80!black,
  colback=yellow!20,
  #1
}


\newtcolorbox{bluebox}[1][]
{%
  enhanced,
  %center upper,
  drop fuzzy shadow southeast,
  boxrule=0pt,
  %sharp corners,
  colframe=blue!10,
  colback=blue!10,
  #1
}

%%%%%%%%%%%%%%%%%%%%%%%%%%%%%%%%%%%%%%%%%%%%%%%%%%%%%%%%%%%%%%
%% Macros

\newcommand{\embf}[1]{\emph{\textbf{#1}}}
\newcommand{\mhask}[1]{\text{\mintinline{Haskell}{#1}}}
\newmintinline[haskell]{haskell}{fontsize=\small}
\newmintinline[coq]{coq}{fontsize=\small}
\newmintinline[ocaml]{OCaml}{fontsize=\small}
\newminted{coq}{fontsize=\small}
\newminted{ocaml}{fontsize=\small}
\newminted{haskell}{fontsize=\small}


\newcommand\blfootnote[1]{%
  \begingroup
  \renewcommand\thefootnote{}\footnote{#1}%
  \addtocounter{footnote}{-1}%
  \endgroup
}


\newcommand\aWrap[1]{#1}
\newcommand\aConst[1]{\kw{const} \; #1}
\newcommand\aId{\kw{id}}
\newcommand\aComp{\mathbin{\circ}}
\newcommand\aProj[1]{\pi_{#1}}
\newcommand\aSplit{\mathbin{\vartriangle}}
\newcommand\aInj[1]{\kw{inj}_{#1}}
\newcommand\aCase{\mathbin{\triangledown}}
\newcommand\aIn{\kw{in}}
\newcommand\aOut{\kw{out}}
\newcommand\aRec{\kw{rec}}

\newcommand\mSend{\mathtt{send}}
\newcommand\mRecv{\mathtt{recv}}
\newcommand\mRet{\mathtt{return}}
\newcommand\mBind{\mathbin{{\tt >}\!\!{\tt >}\!{\tt =}}}
\newcommand\mComp{\mathbin{{\tt >}\!\!{\tt =}\!\!{\tt >}}}
\newcommand\mPar{\mathbin{\wedge}}
\newcommand\mChoice{\mathtt{choice}}
\newcommand\mBranch{\mathtt{branch}}

\newcommand\WF{\mathsf{WF}}
\newcommand\pexp{\text{\color{blue}{\tt e}}}
\newcommand\Gen[1]{\left\llbracket #1 \right\rrbracket}
\newcommand\AT{\text{\tt @}}

\newcommand\lSend[1]{!\langle #1 \rangle}
\newcommand\lRecv[1]{?(#1)}


\definecolor{dgreen}{RGB}{0,160,0}

\renewcommand\UrlFont{\color{red}\rmfamily\itshape}

\newcommand\vc[1]{\mathsf{#1}}

\usepackage{bm}
  \newcommand\role[1]{\mathsf{\bm{#1}}}
  \newcommand\kw[1]{\mathtt{#1}}
  \newcommand\kww[1]{\bm{\mathsf{#1}}}
  \newcommand\size[1]{\mathsf{#1}}
  \newcommand\crecv{c_{\mathsf{i}}}
  \newcommand\cost{C}
  \newcommand\Time{\mathcal{T}}
  \newcommand\Queue{\mathcal{Q}}
  \newcommand\addcost{\hookrightarrow}
  \newcommand\hasCost{\diamond}
  \newcommand\csend{c_{\mathsf{o}}}
  \newcommand\gMsgt{\rightsquigarrow}
  \newcommand\gMsg{\to}
  \newcommand\gFix{\mu}
  \newcommand\gEnd{\kw{end}}
  \newcommand\act{\alpha}
  \newcommand\unroll{\mathsf{unfold}}
  \newcommand\tRun{{\diamond}}
  % \newcommand{\rulename}[1]{\lbrack \text{#1}\rbrack}
  \newcommand\tSend{\mathbin{!}}
  \newcommand\tRecv{\mathbin{?}}
  \newcommand\tSelect{\mathbin{\oplus}}
  \newcommand\tBranch{\mathbin{\&}}
  \newcommand\gEval{\tRun}
  \newcommand\gTy[1]{ : \{#1\}}

%%%%%%%%%%%%%%%%%%%%%%%%%%%%%%%%%%%%%%%%%%%%%%%%%%%%%%%%%%%%%%

\EditInstance{footer}{std}{
  color         = black!50,
  separator     = \hfill,
  element-order = {date, author, title, institute, framenumber, totalframes},
  right-hspace  = 3mm,
  left-hspace   = 3mm,
}
\title{A Synthetic Reconstruction of Multiparty Session Types}
\author{David Castro-P\'erez}
\date{PLAS Seminar, 19-01-2026}

\pagecolor{lightgray!30}
%\color{white}
\begin{document}

\begin{frame}
    \begin{center}
    \Huge
    A Synthetic Reconstruction of Multiparty Session Types        

    \vspace{1cm}

    {\large
    \begin{tabular}{l @{\hspace{3em}} l @{\hspace{3em}} l}
    \underline{David Castro-P\'erez} & Francisco Ferreira & Sung-Shik Jongmans \\
    \scriptsize\href{mailto:d.castro-perez@kent.ac.uk}{\ttfamily d.castro-perez@kent.ac.uk} &
    \scriptsize\href{mailto:francisco.ferreiraruiz@rhul.ac.uk}{\ttfamily francisco.ferreiraruiz@rhul.ac.uk} &
    \scriptsize\href{mailto:s.s.t.q.jongmans@rug.nl}{\ttfamily s.s.t.q.jongmans@rug.nl} \\[.1cm]
    \hspace{2em}\includegraphics[width=4em]{images/KentLogo.png} & 
    \hspace{3em}\includegraphics[width=4em]{images/RHULLogo.png} & 
    \hspace{2.4em}\includegraphics[width=4em]{images/RUGLogo.png} \\
    \end{tabular}
    }

    \end{center}

    \vspace{1cm}

    {\small 
    PLAS Seminar, 19-01-2026}
\end{frame}

\begin{frame}
    \begin{columns}
    \begin{column}{0.35\textwidth}
    \begin{sticky}
    \textbf{Concurrency is hard!}
    \begin{itemize}
    \item \underline{Deadlocks}
    \item \underline{Protocol violations}
    \item Resource contention
    \item etc.
    \end{itemize}
    \end{sticky}
    \end{column}
    \hspace{1cm}

    \begin{column}{0.58\textwidth}
    \begin{sticky}
    \textbf{Our work:}
    \begin{itemize}
    \item Safety and liveness of message-passing concurrent programs
    \item A novel Multiparty Session Type system
    \item Full Agda mechanisation
    \item An implementation in Rascal 
    \end{itemize}
    \end{sticky}
    \end{column}
\end{columns}
\end{frame}

\begin{frame}
    \frametitle{The Problem}
    \LARGE

    \begin{alltt}
    P := send Q; ... 

    Q := receive P; \only<1-2>{receive S}\only<3>{{\color{red!80!black}receive S}}; send R; \only<1-2>{send S}\only<3>{{\color{red!80!black}send S}}; ... 
    
    R := receive Q; send S; ... 
    
    S := receive R;  \only<1-2>{receive Q}\only<3>{{\color{red!80!black}receive Q}}; \only<1-2>{send Q}\only<3>{{\color{red!80!black}send Q}}; ...
    
    system := P | Q | R | S 
    \end{alltt}
    \begin{uncoverenv}<2->
        \begin{center}
        {\textbf{Question:} Is \texttt{system} safe?} \only<3>{\color{red!80!black}\Large\bfseries NO!}
        \end{center}
    \end{uncoverenv}
\end{frame}


\begin{frame}
    \frametitle{Multiparty Session Types (in a nutshell)}
    \begin{center}
    \begin{onlyenv}<1>
        \includegraphics[width=.8\textwidth,page=1]{images/MPST.pdf}
    \end{onlyenv}
    \begin{onlyenv}<2>
        \includegraphics[width=.8\textwidth,page=2]{images/MPST.pdf}
    \end{onlyenv}
    \end{center}
\end{frame}

\begin{frame}
    \frametitle{MPST in more detail}
    \[\begin{array}{l l l l l}
      \text{Roles} & & & \Rp, \Rq, \ldots & \\\\
      \text{Sorts} & \expt{S} & \mathrel{\coloneqq} & \sbool \mid \snat \mid \ldots & \text{Basic data types.} \\\\
      \text{Global Types} & \globalt{G} & \mathrel{\coloneqq} &
      \gmsg{\Rp}{\Rq}{{\gmsgbr{\lbl_i}{S_i}{G_i}}_{i \in I}} & \text{Communication.} \\
      & & \mid & \grecur{X}{G} & \text{Recursion.} \\
      & & \mid & \pvar{X} & \text{Variable.} \\
      & & \mid & \gfinish & \text{End of protocol.} \\\\
      \text{Local Types} & \localt{L} & \mathrel{\coloneqq} &
      \lsend{\Rp}{{\lmsgbr{\lbl_i}{S_i}{L_i}}_{i \in I}} & \text{Send.} \\
      & & \mid & \lrecv{\Rq}{{\lmsgbr{\lbl_i}{S_i}{L_i}}_{i \in I}} & \text{Receive.} \\
      & & \mid & \lrecur{X}{L} & \text{Recursion.} \\
      & & \mid & \pvar{X} & \text{Recursion variable.} \\
      & & \mid & \lfinish & \text{End of protocol.}
    \end{array}
    \]
\end{frame}

\begin{frame}\frametitle{Projection}

\begin{displaymath}
  \begin{array}{c}
\gproj{\gmsg{\Rp}{\Rq}{{\gmsgbr{\lbl_i}{S_i}{G_i}}_{i\in I}}}{\Rr} =
\left\{%
\begin{array}{l l}
\lsend{\Rq}{{\lmsgbr{\lbl_i}{S_i}{\gproj{G_i}{\Rr}}}_{i\in I}} & (\Rr = \Rp \wedge \phantom{\Rr = \Rq} \wedge \Rp \neq \Rq) \\
\lrecv{\Rp}{{\lmsgbr{\lbl_i}{S_i}{\gproj{G_i}{\Rr}}}_{i\in I}} & (\phantom{\Rr = \Rp} \wedge \Rr = \Rq \wedge \Rp \neq \Rq) \\
\lmerge_{i \in I}(\gproj{G_i}{\Rr}) & (\Rr \neq \Rp \wedge \Rr \neq \Rq \wedge \Rp \neq \Rq) \\
\end{array}
\right.
\\[1cm]
\gproj{\grecur{X}{G}}{\Rr} =
\left\{%
\begin{array}{l l}
  \lrecur{X}{\gproj{G}{\Rr}} & (\Rr \in G) \\
  \lfinish & (\Rr \not\in G)
\end{array}
\right.
\hspace{1cm}
\gproj{\pvar{X}}{\Rr} = \pvar{X}
\hspace{1cm}
\gproj{\gfinish}{\Rr} = \lfinish
  \end{array}
\end{displaymath}

\noindent\makebox[\linewidth]{\rule{\columnwidth}{0.4pt}}

\uncover<2->{%
\begin{displaymath}
  \begin{array}{c}
    \begin{array}{l}
\lrecv{\Rp}{{\lmsgbr{\lbl_i}{S_i}{L_i}}_{i\in I}}
\mathbin{\lmerge}
\lrecv{\Rp}{{\lmsgbr{\lbl_j}{S_j}{L'_j}}_{j\in J}}
\\\quad
= \lrecv{\Rp}{{\lmsgbr{\lbl_i}{S_i}{L_i}}_{i\in I\setminus J}
  \cup {\lmsgbr{\lbl_j}{S_j}{L'_j}}_{j\in J\setminus I}
\cup {\lmsgbr{\lbl_i}{S_i}{L_i \mathbin{\lmerge} L'_i}}_{i\in I \cap J}}
\\[.5cm]
\lsend{\Rp}{{\lmsgbr{\lbl_i}{S_i}{L_i}}_{i\in I}}
\mathbin{\lmerge}
\lsend{\Rp}{{\lmsgbr{\lbl_i}{S_i}{L'_i}}_{i\in I}}
=
\lsend{\Rp}{{\lmsgbr{\lbl_i}{S_i}{L_i \lmerge L'_i}}_{i\in I}}
\\[.5cm]
\lrecur{X}{L} \mathbin{\lmerge} \lrecur{X}{L'} = \lrecur{X}{(L \mathbin{\lmerge} L')}
\hspace{1cm}
L \mathbin{\lmerge} L = L
\end{array}
  \end{array}
\end{displaymath}
}
  \Put(40,290){%
    \begin{onlyenv}<3>
    \begin{minipage}{.86\columnwidth}
    \begin{sticky}
      \Large
      \textbf{It gets complicated very quickly!}
    \end{sticky}
    \end{minipage}
    \end{onlyenv}
  }
\end{frame}


\begin{frame}\frametitle{Projection (Example)}
  Consider the following protocol

  \begin{displaymath}
    \grecur{X}{%
\gmsg{\Rp}{\Rq}{\left\{%
  \begin{array}{l@{}l}
   \dlbl{\mathsf{REQ}}(\snat) &. \gmsg{\Rq}{\Rr}{\dlbl{\mathsf{REQ}}(\sbool). \pvar{X}} \\
   \dlbl{\mathsf{END}}() &. \gmsg{\Rq}{\Rr}{\dlbl{\mathsf{END}}(). \pfinish}
  \end{array}
   \right\}}}
  \end{displaymath}

  \vspace{1cm}
  \uncover<2->{%
    Projecting $\Rr$
  \[
    \begin{array}{l}
    \lrecur{X}{%
      (\lrecv{\Rq}{\dlbl{\mathsf{REQ}}(\sbool). \pvar{X}})
        \mathbin{\lmerge}
      (\lrecv{\Rq}{\dlbl{\mathsf{END}}(). \lfinish})
   }
    \\[.3cm] \qquad =
    \uncover<3>{%
    \lrecur{X}{%
      \lrecv{\Rq}{\left\{%
  \begin{array}{l@{}l}
   \dlbl{\mathsf{REQ}}(\sbool) &. \pvar{X} \\
   \dlbl{\mathsf{END}}() &. \pfinish
  \end{array}
   \right\}}
    }}
    \end{array}
  \]
%   }
  }
\end{frame}



\begin{frame}\frametitle{Processes and Typing}
\begin{displaymath}
  \begin{array}{lrcll}
    \text{Process} & \proc{P} & \mathrel{:=} & \psend{p}{\lbl}{e}{P} & \text{Send a message.} \\
    & & \mid & \precv{i}{I}{\precbr{p}{\lbl_i}{x_i}{P_i}} & \text{Receive a message.} \\
    & & \mid & \pif{e}{P}{P'} & \text{Conditional process.} \\
    & & \mid & \precur{X}{P} & \text{Recursive process.} \\
    & & \mid & \pvar{X} & \text{Recursion variable.} \\
    & & \mid & \pfinish & \text{Inactive process.} \\
  \end{array}
\end{displaymath}

\end{frame}

\begin{frame}\frametitle{Process Typing (simplified)}
  Once we have local types, process typing is simple:
  \vspace{.4cm}

\[
  \begin{array}{c}
    \infer[T-SEND]{%
      \loft{\Gamma}{P}{L_i} \and \toft{\Gamma}{e}{S_i} \and i \in I
    }{%
      \loft{\Gamma}{\psend{\Rq}{\lbl_i}{e}{P}}{(\lsend{\Rp}{{\lmsgbr{\lbl_i}{S_i}{L_i}}_{i\in I}})}
    }
    %
    \hspace{1cm}
    \infer[T-RECV]{%
      \forall (i \in I), \left[\loft{\Gamma ,{\pvar{x_i}}:{\expt{S_i}}}{P_i}{L_i}\right]z
    }{%
      \loft{\Gamma}{\precv{i}{I}{\precbr{p}{\lbl_i}{x_i}{P_i}}}{(\lrecv{\Rp}{{\lmsgbr{\lbl_i}{S_i}{L_i}}_{i\in I}})}
    }
  \end{array}
\]
\Put(20,250){
  \begin{onlyenv}<2->
    \begin{minipage}{.55\columnwidth}
      \begin{sticky}
        \large
        Local types and processes are so similar that some developments omit
them, and projection produces directly processes.
      \end{sticky}
    \end{minipage}
  \end{onlyenv}
}
\Put(40,200){
  \begin{onlyenv}<3>
    \begin{minipage}{.75\columnwidth}
      \begin{sticky}
        \includegraphics[width=\columnwidth]{images/deconfined-session-types.pdf}
      \end{sticky}
    \end{minipage}
  \end{onlyenv}
}
\end{frame}

\begin{frame}\frametitle{Problems with Classical Formulation}
  \begin{enumerate}
    \item \textbf{Too syntactic:}
      \begin{itemize}
        \item Processes and local types must align
        \item Too restrictive, rules out correct processes
        \item \ldots
      \end{itemize}
    \item \textbf{Unnecessarily complex:}
      \begin{itemize}
        \item Hard to implement/mechanise, e.g.:
          \begin{itemize}
            \item Use of runtime coinductive global types: Our PLDI 2021 paper, Jacobs et al. (2022).
            \item Graph-based reasoning and decision procedure for the equality
              of recursive types: Tirore et al. (2023)
          \end{itemize}
        \item Hard to extend
      \end{itemize}
    \item \textbf{Imprecise} (coinduction, safety)
  \end{enumerate}
\end{frame}

\begin{frame}\frametitle{Example of Imprecision in Classical MPST}
  \boxed{\text{Equirecursion: ``We identify $\grecur{X}{G}$ with $[\grecur{X}{G} / \pvar{X}] \globalt{G}$''}}

  \vspace{.4cm}

  Common statement in proofs about MPST, but...
  \begin{enumerate}
    \item The rules specify how to deal with variables $\pvar{X}$
    \item The rules specify when and how to unfold $\grecur{X}{G}$
  \end{enumerate}

  \vspace{.4cm}
  Moreover: Equirecursion alone distinguishes too many protocols that ``are the same'':
  \[
    \gmsg{\Rp}{\Rq}{\gmsg{\Rp'}{\Rq'}{G}} \neq \gmsg{\Rp'}{\Rq'}{\gmsg{\Rp}{\Rq}{G}}
  \]

  \vspace{.4cm}
  Mechanising the classical theory of MPST is notoriously hard, in part due to this.

\end{frame}


\begin{frame}\frametitle{Another Example of Imprecision in Classical MPST}
  This source of imprecision \textbf{did} cause flawed proofs in the literature.

  \vspace{1cm}
  \textbf{Preservation theorem:}
  \[
  \begin{array}{l}
       \varphi (\localt{L_1}, \ldots, \localt{L_n}) \wedge (\proc{P_1} \mid \ldots \mid \proc{P_n} \xrightarrow{\alpha} \proc{P_1'} \mid \ldots \mid \proc{P_n'}) \\
       \wedge (\loft{}{P_1}{\localt{L_1}} \wedge \ldots \wedge \loft{}{P_n}{\localt{L_n}})
  \end{array}
  \quad\Longrightarrow\quad
  \begin{array}{l}
       \exists \localt{L_1}' \cdots \localt{L_n}' \wedge (\proc{L_1} \mid \ldots \mid \proc{L_n} \xrightarrow{\alpha} \proc{L_1'} \mid \ldots \mid \proc{L_n'}) \\
       \wedge (\loft{}{P_1'}{\localt{L_1}'} \wedge \ldots \wedge \loft{}{P_n'}{\localt{L_n}'})
  \end{array}
  \]
  \vspace{1cm}
  \begin{uncoverenv}<2->
  Implicit assumption:
  \[
  (\forall i, \localt{L_i} = \gproj{\globalt{G}}{\role{r_i}}) \Longrightarrow \varphi (\localt{L_1}, \ldots, \localt{L_n})
  \]
  \end{uncoverenv}
  \Put(40,200){
  \begin{onlyenv}<3>
    \begin{minipage}{.75\columnwidth}
      \begin{sticky}
        This assumption is \textbf{wrong} (Scalas \& Yoshida, POPL'19)!
        \begin{itemize}
          \item Trivially holds for the basic case.
          \item Breaks as soon as you extend the theory slightly (e.g. full merge).
        \end{itemize}
      \end{sticky}
    \end{minipage}
  \end{onlyenv}
}
\end{frame}

\begin{frame}\frametitle{A Few Attempts at Simplifying the Theory}

  \begin{center}
    \begin{minipage}{.86\columnwidth}
    \begin{sticky}
        \includegraphics[width=\columnwidth]{images/deconfined-session-types.pdf}
    \end{sticky}
    \end{minipage}
  \end{center}

  \Put(40,210){%
    \begin{onlyenv}<2->
    \begin{minipage}{.86\columnwidth}
    \begin{sticky}
  \includegraphics[width=\textwidth]{images/less-is-more.pdf}
    \end{sticky}
    \end{minipage}
    \end{onlyenv}
   }
\Put(60,190){
  \begin{onlyenv}<3->
    \begin{minipage}{.75\columnwidth}
      \begin{sticky}
  \includegraphics[clip, width=\textwidth, trim= 0 18cm 0 0]{images/less-is-more-revisited.pdf}
      \end{sticky}
    \end{minipage}
  \end{onlyenv}
}

\end{frame}


\begin{frame}\frametitle{Our Approach: Synthetic Typing}
    \begin{minipage}{.86\columnwidth}
    \begin{sticky}
  \includegraphics[width=\textwidth]{images/fs-synthetic.pdf}
    \end{sticky}
    \end{minipage}

  \Put(40,140){%
    \begin{onlyenv}<2->
    \begin{minipage}{.86\columnwidth}
      \begin{sticky}
        \textbf{\underline{Our Contributions}:}
        \begin{itemize}
          \item ``Free'' typing from being tied up to the syntax of local types.
          \item An MPST system that avoids projection/merging/etc.
          \item Type-checking against arbitrary (well-formed) LTSs.
          \item Well-formedness/deadlock-freedom is decided by typeability, not by projectability.
          \item Mechanisation in Agda.
          \item Implementation in Rascal.
        \end{itemize}
      \end{sticky}
    \end{minipage}
    \end{onlyenv}
   }
\end{frame}

\begin{frame}\frametitle{(Slightly Simplified) Core SyntheticTyping Rules}
\[
  \infer[T-SEND]
  {%
  \poft{\Gamma}{P}{G'}{\Rp} \and 
  \greduce{G}{\annlbl{\Rp}{\Rq}{\lbl(\expt{S})}}{G'} \and
  \toft{\Gamma}{e}{S}
  }{%
    \poft{\Gamma}{\psend{\Rq}{\lbl}{e}{P}}{G}{\Rp}
  }
\]
\begin{displaymath}
  \infer[T-RECV]
    { \text{($\globalt{G}$ allows $\Rp \to \Rq : \lbl_j$, for some j)}
      \\ \\
      \forall\; i\;\globalt{G}'\;(\greduce{G}{\annlbl \Rq \Rp {\lbl_i(\expt{S_i})}}{G'});
      \left[\poft{\Gamma, {\pvar{x_i}} : {\expt{S_i}}}{P_i}{G'}{\Rp}\right]
  }
  {\poft \Gamma {\precv{i}{I}{\precbr q {\lbl_i} {x_i} {P_i}}}{G}{\Rp}}
\end{displaymath}

\begin{displaymath}
  \infer[T-SKIP]
    {  \text{(It is safe for $\Rp$ to wait in $\globalt{G}$)}
      \and
      \forall (\gredskip{G}{\Rp}{G'}); \left[\poft{\Gamma}{P}{G'}{\Rp}\right]
  }
  {\poft \Gamma P G \Rp}
\end{displaymath}
\end{frame}

\begin{frame}\frametitle{Synthetic Behavioural Typing}
  \Large 

  \begin{center}
  \textbf{Key idea:}
    The syntax of $\globalt{G}$ is irrelevant!
  
    \vspace{2cm}
  \uncover<2->{%
  $\globalt{G}$ is just the state of a \textbf{labelled transition system (LTS)}!
  }
  \end{center}
\end{frame}

\begin{frame}\frametitle{Well-behavedness}
  Our type system is parameterised by an LTS, where labels must specify send/receive interactions.

  \vspace{.4cm}
  \uncover<2->{%
  \alert{But not all LTSs are valid types!} We can only guarantee safety/liveness for \textbf{well-behaved} LTSs.
  }

  \vspace{.4cm}

  \uncover<3->{%
  \textbf{Well-behavedness:}
  \begin{enumerate}
    \item \textbf{Sender determinacy}: If a state allows multiple transitions, these cannot have the same receiver but different senders.
    \item \textbf{Determinism}: A state can have at most one transition with the same action label.
    \item \textbf{Conditional commutativity}: In any state, if a later independent action has an earlier enabled branch, it can be commuted earlier.
    \item \textbf{Diamond property}.
    \item \textbf{"Stepback" property?} (not in the paper -- likely an artifact of our mechanisation only used to prove one minor case in our mechanisation). ``Bisimilarity is preserved when moving to past states'' (?)
  \end{enumerate}  
  }
  \Put(40,220){%
    \begin{onlyenv}<4>
    \begin{minipage}{.86\columnwidth}
    \begin{sticky}
      \Large
      \textbf{Every syntactic global type in the classical theory of MPST is well-behaved!}
    \end{sticky}
    \end{minipage}
    \end{onlyenv}
  }
\end{frame}


\begin{frame}\frametitle{Example}

  \begin{displaymath}
    \globalt{G} = \grecur X {
\gmsg \Rp \Rq {\left\{%
  \begin{array}{l@{}l}
   \dlbl{\mathsf{REQ}}(\snat) &. \gmsg \Rq \Rr {\dlbl{\mathsf{REQ}}(\sbool). \pvar{X}} \\
   \dlbl{\mathsf{END}}() &. \gmsg \Rq \Rr {\dlbl{\mathsf{END}}(). \pfinish}
  \end{array}
   \right\}}}
  \end{displaymath}

  We are going to typecheck a process implementing role $\Rr$...

  \uncover<2>{\textbf{but first, let's get rid of the syntax for $\globalt{G}$!}}
\end{frame}

\begin{frame}\frametitle{Example: Semantic View of Global Types}

  \begin{displaymath}
    \grecur X {
\gmsg \Rp \Rq {\left\{%
  \begin{array}{l@{}l}
   \dlbl{\mathsf{REQ}}(\snat) &. \gmsg \Rq \Rr {\dlbl{\mathsf{REQ}}(\sbool). \pvar{X}} \\
   \dlbl{\mathsf{END}}() &. \gmsg \Rq \Rr {\dlbl{\mathsf{END}}(). \pfinish}
  \end{array}
   \right\}}}
  \end{displaymath}

  \vspace{.3cm}

  \begin{center}
  \begin{tikzpicture}
\node[state, initial] (1) {$1$};
\node[state, right=3cm of 1] (2) {$2$};
\node[state, below=1.5cm of 1] (4) {$4$};
\node[state, right=3cm of 4, accepting] (5) {$5$};
\draw[->]
(1) edge[bend right, below] node{$\annlbl \Rp \Rq {\dlbl{\mathsf{REQ}}(\snat)}$} (2)
(1) edge[left] node{$\annlbl \Rp \Rq {\dlbl{\mathsf{END}}()}$} (4)
(2) edge[bend right, above] node{$\annlbl \Rq \Rr {\dlbl{\mathsf{REQ}}(\sbool)}$} (1)
(4) edge[below] node{$\annlbl \Rq \Rr {\dlbl{\mathsf{END}}()}$} (5);
\end{tikzpicture}
\end{center}
\end{frame}


\begin{frame}[t]\frametitle{Example: Process \& Typing}
  \Put(320,130){%
    \begin{minipage}{.32\columnwidth}
    \begin{bluebox}
      \vspace{-.3cm}
      \resizebox{\columnwidth}{!}{%
        \begin{tikzpicture}[
  ->, % makes the edges directed
  >=stealth', % makes the arrow heads bold
  node distance=3cm, % specifies the minimum distance between two nodes. Change if necessary.
  every state/.style={thick, fill=gray!10}, % sets the properties for each ’state’ node
  initial text=$ $ % sets the text that appears on the start arrow
          ]
\node[state,
onslide=<2>{highlight},
onslide=<3>{highlight},
onslide=<8>{highlight},
onslide=<9>{highlight},
onslide=<10>{highlight},
onslide=<13>{highlight},
onslide=<14>{highlight},
initial] (1) {$1$};
\node[state,
onslide=<4>{highlight},
onslide=<7>{highlight},
onslide=<12>{highlight},
right=3cm of 1] (2) {$2$};
\node[state,
onslide=<4>{highlight},
onslide=<5>{highlight},
onslide=<11>{highlight},
below=.8cm of 1] (4) {$4$};
\node[state,
onslide=<6>{highlight},
onslide=<11>{highlight},
right=3cm of 4, accepting] (5) {$5$};
\draw[->]
(1) edge[bend right, below] node{$\annlbl \Rp \Rq {\dlbl{\mathsf{REQ}}(\snat)}$} (2)
(1) edge[left] node{$\annlbl \Rp \Rq {\dlbl{\mathsf{END}}()}$} (4)
(2) edge[bend right, above] node{$\annlbl \Rq \Rr {\dlbl{\mathsf{REQ}}(\sbool)}$} (1)
(4) edge[below] node{$\annlbl \Rq \Rr {\dlbl{\mathsf{END}}()}$} (5);
\end{tikzpicture}
}\vspace{-.3cm}
    \end{bluebox}
    \end{minipage}
  }

  \vspace{1.5cm}

  \only<1>{%
    $
    \proc{P} = \proc{\sum}\left\{
    \begin{array}{@{}l@{}}
    \precbr \Rq {\dlbl{\mathsf{REQ}}} {x} {\mathsf{print}(\pvar{x})\proc{.}\;
      \precur X {%
\sum\left\{
    \begin{array}{@{}l@{}}
    \precbr \Rq {\dlbl{\mathsf{REQ}}} {x} {\mathsf{process}(\pvar{x})\proc{.}\;
      \pvar{X}
      }
      \\
    \precbr \Rq {\dlbl{\mathsf{END}}} {\_} {\pfinish}
    \end{array}\right\}
      }
      }
      \\
    \precbr \Rq {\dlbl{\mathsf{END}}} {\_} {\pfinish}
    \end{array}\right\}$
  }
  \only<2-4>{%
    \colorbox{orange!30}{%
    $
    \proc{P} = \proc{\sum}\left\{
    \begin{array}{@{}l@{}}
    \precbr \Rq {\dlbl{\mathsf{REQ}}} {x} {\mathsf{print}(\pvar{x})\proc{.}\;
      \precur X {%
\sum\left\{
    \begin{array}{@{}l@{}}
    \precbr \Rq {\dlbl{\mathsf{REQ}}} {x} {\mathsf{process}(\pvar{x})\proc{.}\;
      \pvar{X}
      }
      \\
    \precbr \Rq {\dlbl{\mathsf{END}}} {\_} {\pfinish}
    \end{array}\right\}
      }
      }
      \\
    \precbr \Rq {\dlbl{\mathsf{END}}} {\_} {\pfinish}
    \end{array}\right\}
    $
    }
  }
  \only<5>{%
    $
    \proc{P} = \proc{\sum}\left\{
    \begin{array}{@{}l@{}}
    \precbr \Rq {\dlbl{\mathsf{REQ}}} {x} {\mathsf{print}(\pvar{x})\proc{.}\;
      \precur X {%
\sum\left\{
    \begin{array}{@{}l@{}}
    \precbr \Rq {\dlbl{\mathsf{REQ}}} {x} {\mathsf{process}(\pvar{x})\proc{.}\;
      \pvar{X}
      }
      \\
    \precbr \Rq {\dlbl{\mathsf{END}}} {\_} {\pfinish}
    \end{array}\right\}
      }
      }
      \\
    \colorbox{orange!30}{\precbr \Rq {\dlbl{\mathsf{END}}} {\_} {\pfinish}}
    \end{array}\right\}
    $
    }
    \only<6>{%
    $
    \proc{P} = \proc{\sum}\left\{
    \begin{array}{@{}l@{}}
    \precbr \Rq {\dlbl{\mathsf{REQ}}} {x} {\mathsf{print}(\pvar{x})\proc{.}\;
      \precur X {%
\sum\left\{
    \begin{array}{@{}l@{}}
    \precbr \Rq {\dlbl{\mathsf{REQ}}} {x} {\mathsf{process}(\pvar{x})\proc{.}\;
      \pvar{X}
      }
      \\
    \precbr \Rq {\dlbl{\mathsf{END}}} {\_} {\pfinish}
    \end{array}\right\}
      }
      }
      \\
    \precbr \Rq {\dlbl{\mathsf{END}}} {\_} {\colorbox{orange!30}{\pfinish}}
    \end{array}\right\}
    $
    }
    \only<7>{%
    $
    \proc{P} = \proc{\sum}\left\{
    \begin{array}{@{}l@{}}
    \colorbox{orange!30}{\precbr \Rq {\dlbl{\mathsf{REQ}}} {x} {\mathsf{print}(\pvar{x})\proc{.}\;
      \precur X {%
\sum\left\{
    \begin{array}{@{}l@{}}
    \precbr \Rq {\dlbl{\mathsf{REQ}}} {x} {\mathsf{process}(\pvar{x})\proc{.}\;
      \pvar{X}
      }
      \\
    \precbr \Rq {\dlbl{\mathsf{END}}} {\_} {\pfinish}
    \end{array}\right\}
      }
      }}
      \\
    \precbr \Rq {\dlbl{\mathsf{END}}} {\_} {\pfinish}
    \end{array}\right\}
    $
    }
    \only<8,9>{%
    $
    \proc{P} = \proc{\sum}\left\{
    \begin{array}{@{}l@{}}
    \precbr \Rq {\dlbl{\mathsf{REQ}}} {x} {\mathsf{print}(\pvar{x})\proc{.}\;
      \colorbox{orange!30}{\precur X {%
\sum\left\{
    \begin{array}{@{}l@{}}
    \precbr \Rq {\dlbl{\mathsf{REQ}}} {x} {\mathsf{process}(\pvar{x})\proc{.}\;
      \pvar{X}
      }
      \\
    \precbr \Rq {\dlbl{\mathsf{END}}} {\_} {\pfinish}
    \end{array}\right\}
      }
      }}
      \\
    \precbr \Rq {\dlbl{\mathsf{END}}} {\_} {\pfinish}
    \end{array}\right\}
    $
    }
  \only<10>{%
    $
    \proc{P} = \proc{\sum}\left\{
    \begin{array}{@{}l@{}}
    \precbr \Rq {\dlbl{\mathsf{REQ}}} {x} {\mathsf{print}(\pvar{x})\proc{.}\;
      \precur X {%
\text{\colorbox{orange!30}{$\sum\left\{
    \begin{array}{@{}l@{}}
    \precbr \Rq {\dlbl{\mathsf{REQ}}} {x} {\mathsf{process}(\pvar{x})\proc{.}\;
      \pvar{X}
      }
      \\
    \precbr \Rq {\dlbl{\mathsf{END}}} {\_} {\pfinish}
    \end{array}\right\}$
      }
      }}}
      \\
    \precbr \Rq {\dlbl{\mathsf{END}}} {\_} {\pfinish}
    \end{array}\right\}
    $
    }
  \only<11>{%
    $
    \proc{P} = \proc{\sum}\left\{
    \begin{array}{@{}l@{}}
    \precbr \Rq {\dlbl{\mathsf{REQ}}} {x} {\mathsf{print}(\pvar{x})\proc{.}\;
      \precur X {%
\sum\left\{
    \begin{array}{@{}l@{}}
    \precbr \Rq {\dlbl{\mathsf{REQ}}} {x} {\mathsf{process}(\pvar{x})\proc{.}\;
      \pvar{X}
      }
      \\
      \colorbox{orange!30}{
    \precbr \Rq {\dlbl{\mathsf{END}}} {\_} {\pfinish}
      }
    \end{array}\right\}
      }
      }
      \\
    \precbr \Rq {\dlbl{\mathsf{END}}} {\_} {\pfinish}
    \end{array}\right\}
    $
    }
    \only<12>{%
    $
    \proc{P} = \proc{\sum}\left\{
    \begin{array}{@{}l@{}}
    \precbr \Rq {\dlbl{\mathsf{REQ}}} {x} {\mathsf{print}(\pvar{x})\proc{.}\;
      \precur X {%
\sum\left\{
    \begin{array}{@{}l@{}}
            \colorbox{orange!30}{%
    \precbr \Rq {\dlbl{\mathsf{REQ}}} {x} {\mathsf{process}(\pvar{x})\proc{.}\;
      \pvar{X}
      }}
      \\
    \precbr \Rq {\dlbl{\mathsf{END}}} {\_} {\pfinish}
    \end{array}\right\}
      }
      }
      \\
    \precbr \Rq {\dlbl{\mathsf{END}}} {\_} {\pfinish}
    \end{array}\right\}
    $
    }
  \only<13,14>{%
    $
    \proc{P} = \proc{\sum}\left\{
    \begin{array}{@{}l@{}}
    \precbr \Rq {\dlbl{\mathsf{REQ}}} {x} {\mathsf{print}(\pvar{x})\proc{.}\;
      \precur X {%
\sum\left\{
    \begin{array}{@{}l@{}}
    \precbr \Rq {\dlbl{\mathsf{REQ}}} {x} {\mathsf{process}(\pvar{x})\proc{.}\;
      \colorbox{orange!30}{\pvar{X}
      }}
      \\
    \precbr \Rq {\dlbl{\mathsf{END}}} {\_} {\pfinish}
    \end{array}\right\}
      }
      }
      \\
    \precbr \Rq {\dlbl{\mathsf{END}}} {\_} {\pfinish}
    \end{array}\right\}
    $
    }

  \vspace{.5cm}

  \uncover<3,9->{
    \only<3>{Our goal is to show that $\poft{}{P}{\globalt{1}}{\Rr}$}
    \only<9>{$\poft{\cdot}{\precur{X}{\colorbox{orange!30}{\ldots}}}{1}{\Rr}$}
    \only<10-13>{$\poft{\pvar{X} : \globalt{1}}{\colorbox{orange!30}{\ldots}}{\colorbox{red!50}{\ldots}}{\Rr}$}
        \only<14>{$\poft{\pvar{X} : \globalt{1}}{\colorbox{orange!30}{\pvar{X}}}{\colorbox{red!50}{\globalt{1}}}{\Rr}$}


  }
\end{frame}


\begin{frame}\frametitle{Properties of Synthetic MPST}

  Some key lemmas:
  \begin{displaymath}
    \begin{array}{@{}l@{}}
    \bullet\; \text{If } \globalt{G} \bisim \globalt{G'} \text{ and } \poft {\Gamma} {P} {G} \Rr \text{ then } \poft {\Gamma} {P} {G'} \Rr \\[.3cm]
      \bullet\;
      \text{If } \greduce G \alpha {G'}\text{, with }\Rr \not\in \alpha\text{, and } \poft {\Gamma} {P} {G} \Rr\text{, then } \poft {\Gamma} {P} {G'} \Rr
    \end{array}
  \end{displaymath}

  \vspace{.5cm}

  \uncover<2->{%
  These are needed for proving safety and liveness theorems (i.e. preservation andf progress). Suppose that $\Sm$ is a collection of 
  processes, and $\globalt{G}$ is \alert{well-behaved}:
  \begin{displaymath}
    \begin{array}{@{}l@{}}
      \bullet\; \text{If }\soft \Sm G\text{ and }\Sm \stepsto{\alpha} \Sm'\text{, then there exists }\globalt{G'}\text{ such that }
      \greduce G \alpha {G'}\text{ and }\soft {\Sm'} {G'} \\[.4cm]
      \bullet\; \text{If }\soft \Sm G \text{ and }\globalt{G}\text{ is not ended, then there exists }\Sm'\text{ and }\alpha\text{ such that }\Sm \stepsto{\alpha} \Sm'.
    \end{array}
  \end{displaymath}
  }

  \uncover<3>{%
  \vspace{.5cm}
  Finally, we proved that for all global type $\globalt{G}$, the LTS of $\globalt{G}$ is well-behaved.
  }
\end{frame}

\begin{frame}\frametitle{Contributions}
  \begin{itemize}
  \item Special case: MPST system with global types (without projection, merging, local types)
  \item Special case: Expressive enough to capture all benchmarks of (Scalas \& Yoshida 2019)
  \item[] 
  \item General case: MPST system with \emph{well-behaved LTSs} 
  \item Type soundness -- much simpler than the literature (roughly 550 LOC of Agda!)
  \item[]
  \item Artifact: Full mechanisation in Agda. 
  \item Artifact: Implementation of the special case in Rascal (Thanks, Sung!)
  \end{itemize}
  \Put(80,360){%
    \begin{onlyenv}<2>
    \begin{minipage}{.56\columnwidth}
    \begin{sticky}
      \Huge
      \textbf{THANKS!}
    \end{sticky}
    \end{minipage}
    \end{onlyenv}
  }
\end{frame}



\end{document}